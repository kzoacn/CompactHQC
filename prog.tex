\documentclass[zihao=-4,fontset=fandol,linespread=1.5]{ctexart}
\usepackage{pifont} % 提供\ding{72}五角星指令
\usepackage{fancyhdr} % 自定义页眉页脚
%\usepackage{geometry} % 调整页面边距
%\geometry{left=2in, right=2in, top=2in, bottom=2in}
\usepackage{zhlipsum}
\usepackage{listings}
\usepackage{xcolor}

\lstset{
    basicstyle=\ttfamily\small,
    breaklines=true,
    frame=single,
    numbers=left,
    numberstyle=\tiny\color{gray},
    keywordstyle=\color{blue},
    commentstyle=\color{green},
    stringstyle=\color{red},
    showstringspaces=false,
    tabsize=4
}
\ctexset{section={
format = \heiti\raggedright,
format += \zihao{3},
name={,、},
number=\chinese{section},
}
}

\pagestyle{plain} % 默认的其他页面样式为plain(无页眉)

\begin{document}

% 封面页
\thispagestyle{fancy} % 仅对当前页启用自定义页眉
\fancyhf{} % 清除默认设置
\fancyhead[L]{} % 页眉左侧内容
\fancyhead[R]{\heiti 内部\\ 页数: } % 页眉右侧内容
\renewcommand{\headrulewidth}{0pt} % 隐藏页眉下的线
\setlength{\headheight}{15pt}

\vspace*{\fill} % 页面顶部留空,内容居中

\begin{center}
    \setlength{\lineskip}{1em} %\setlength的影响限于本地
    {\zihao{1} \songti \bfseries {CompactHQC密码算法\\ C语言实现 \\ 1.0版本} } % 主标题
\end{center}

\vspace*{\fill} % 页面底部留空
\vspace*{\fill} % 页面底部留空
\vspace*{\fill} % 页面底部留空

\begin{center}
    {\zihao{3} \heiti 上海交通大学密码学实验室} \\ % 作者姓名
    {\zihao{4} \heiti \today} % 作者单位
\end{center}

\renewcommand{\thepage}{---\arabic{page}---}

\newpage % 新页开始

\setcounter{page}{1}

\tableofcontents
\newpage

% 后续页面内容

\section{程序说明}

CompactHQC实现包含两个安全级别版本(CHQC-320和CHQC-512),具有相同的文件结构,以chqc-512为例:

\subsection{核心算法文件}
\begin{itemize}
    \item \texttt{src/chqc.[ch]} - 主算法实现
    \item \texttt{src/kem.c} - 密钥封装机制
    \item \texttt{src/parameters.h} - 算法参数配置
\end{itemize}

\subsection{数学运算模块}
\begin{itemize}
    \item \texttt{src/gf.[ch]} - 有限域运算
    \item \texttt{src/gf2x.[ch]} - 二元多项式运算
    \item \texttt{src/fft.[ch]} - 快速傅里叶变换
    \item \texttt{src/toom.c} - Toom-Cook乘法
    \item \texttt{src/toom-gpl.c} - GPL授权的Toom-Cook实现
\end{itemize}

\subsection{编码模块}  
\begin{itemize}
    \item \texttt{src/reed\_muller.[ch]} - Reed-Muller编码
    \item \texttt{src/reed\_solomon.[ch]} - Reed-Solomon编码
    \item \texttt{src/code.[ch]} - 编码接口
\end{itemize}

\subsection{辅助模块}
\begin{itemize}
    \item \texttt{src/vector.[ch]} - 向量操作
    \item \texttt{src/parsing.[ch]} - 数据解析
    \item \texttt{src/shake\_prng.[ch]} - 伪随机数生成
    \item \texttt{src/shake\_ds.[ch]} - 确定性采样
\end{itemize}

\subsection{依赖库}
\begin{itemize}
    \item \texttt{lib/fips202/fips202.[ch]} - SHA-3哈希函数实现
\end{itemize}

\subsection{测试程序}
\begin{itemize}
    \item \texttt{src/main\_chqc.c} - 主测试程序 
\end{itemize}

\subsection{使用说明}
\begin{enumerate}
    \item 执行 \texttt{make chqc-512} 编译当前目录版本
    \item 运行测试: \texttt{./bin/chqc-512}
\end{enumerate}

\section{模拟实现程序}

\subsection{核心算法实现 (chqc.c)}
\lstinputlisting[language=C]{chqc-512/src/chqc.c}

\subsection{密钥封装机制 (kem.c)}
\lstinputlisting[language=C]{chqc-512/src/kem.c}

\subsection{有限域运算 (gf.c)}
\lstinputlisting[language=C]{chqc-512/src/gf.c}

\subsection{FFT实现 (fft.c)}
\lstinputlisting[language=C]{chqc-512/src/fft.c}

\subsection{Reed-Muller编码 (reed\_muller.c)}
\lstinputlisting[language=C]{chqc-512/src/reed_muller.c}

\section{算法样本数据}

给出密码算法的输入/输出样本数据


\end{document}
